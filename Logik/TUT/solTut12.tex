\documentclass{article}
\author{Max Springenberg}
\title{Logik TUT 12}
\usepackage{amsmath}
\usepackage{amssymb}
\usepackage{stmaryrd}
\usepackage{mathtools}
\setcounter{section}{12}

\begin{document}
\maketitle
\newpage

\subsection{Prolog}
\subsubsection\
$
M(x)\\
K(x,y)\\
\\
((M(x) \land \neg M(y) \land K(x,y)) \rightarrow V(x,y))\\
$
es sollte nur m (M(x)) und kleiner\_als (K(X,y)) benutzt werden, 
deshalb habe ich w fuer den Vergleich (V(x,y)) entfallen lassen.
\subsubsection\
Tim fragt nur ab, ob es in der Menge an Fakten bereits ein 
Maedchen gibt, mit dem er verglichen kleiner ist.\\
\\
kleiner\_als(X,Z) :- fakt\_kleiner\_als(X,Y), fakt\_kleiner\_als(Y,Z)\\
\subsubsection\
liegt\_zwischen(X,Y,Z) :- kleiner\_als(X,Z), kleiner\_als(Z,Y)\\
\subsubsection\
? :- kleiner\_als(Ben,Z)\\
Mir fehlt jegliche Motivation einen gesamten Baum mit LaTeX zu schreiben.\\
\newpage
\subsection{Mehr Prolog}
\begin{tabular}{l|l}
        liegen\_zwischen(X,Y,Z) :- liegt\_zwischen(X,Y,Z)   &name\\
        liegen\_zwischen(X,Y,Z) :- liegt\_zwischen(X,Y,Z'),
                        liegen\_zwischen(X,Y,Z')            &name(name,+++(+++,---,(---,......)))\\
\end{tabular}\\
\end{document}
