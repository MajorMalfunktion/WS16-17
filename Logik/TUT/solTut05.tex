\documentclass{article}
\author{Max Springenberg}
\title{Logik TUT 5}
\usepackage{amsmath}
\usepackage{amssymb}
\usepackage{stmaryrd}
\setcounter{section}{5}

\begin{document}
\maketitle
\newpage

\subsection{Noch ein Erfuellbarkeitstest}
\subsubsection\
NNF bilden:\\
$
\varphi =   \neg ((\neg A \lor B) \rightarrow (C \lor \neg A))\\
\equiv      \neg (\neg (\neg A \lor B) \lor (C \lor \neg A))\\
\equiv      (\neg A \lor B) \land \neg (C \lor \neg A)\\
\equiv      (\neg A \lor B) \land \neg C \land A
= \varphi'\\
\\
$Tableaukalkuel:\\$
\begin{matrix}
        &A\\
        &|\\
        &\neg C\\
        &|\\
        &(\neg A \lor B)\\
        &l,r\\
        l:\\
        \neg A\\
        \lightning\\
        &&r:\\
        &&B\\ 
        &&\checkmark\\
\end{matrix}\\
$
\subsubsection\
NNF bilden:\\
$
\varphi =   \neg ((A \rightarrow B) \rightarrow B) \land A\\
\equiv      \neg (\neg(A \rightarrow B) \lor B) \land A\\
\equiv      \neg (\neg(\neg A \lor B) \lor B) \land A\\
\equiv      (\neg A \lor B) \land \neg B \land A
= \varphi'\\
\\
$Tableaukalkuel:\\$
\begin{matrix}
        &A\\
        &|\\
        &\neg B\\
        &|\\
        &(\neg A \lor B)\\
        &l,r\\
        l:\\ 
        \neg A\\ 
        \lightning\\
        &&r:\\ 
        &&B\\ 
        &&\lightning\\
\end{matrix}\\
$
\subsubsection\
NNF vorhanden\\
\\
Tableaukalkuel:\\
$
\begin{matrix}
        &s1, \varphi\\
        &l,r\\
        l: s1, \Diamond (\Diamond A \land B)\\
        |\\
        (s1, s2) \in E\\
        |\\
        s2, (\Diamond A \land B)\\
        |\\
        s2, B\\
        |\\
        s2, \Diamond A\\
        |\\
        (s2, s3) \in E\\
        |\\
        s3, A\\ 
        \checkmark\\
        &&r: s1, \Diamond B\\
        &&|\\
        &&(s1, s4) \in E\\
        &&|\\
        &&s4, B\\ 
        &&\checkmark\\
\end{matrix}\\
$
\subsubsection\
NNF vorhanden\\
\\
Tableaukalkuel:\\
$
\begin{matrix}
        &s1, \varphi\\
        &\Diamond A\\
        &|\\
        &\Box C\\
        &|\\
        &\Diamond(\Box A \land \Box \neg B)\\
        &|\\
        &(s1,s2) \in E\\
        &|\\
        &s2, C\\
        &|\\
        &s2, A\\
        &|\\
        &(s1, s3) \in E\\
        &|\\
        &s3, C\\
        &|\\
        &s3, (\Box A \land \Box \neg B)\\
        &|\\
        &(s3, s4) \in E\\
        &|\\
        &s4, A\\
        &|\\
        &s4, \neg B\\
        &\checkmark\\
\end{matrix}\\
$
\subsubsection\
NNF bilden:\\
$
\varphi =   \Box (B \rightarrow (\Diamond A \land \Box \neg A)) \land \Box B\\
\equiv      \Box (\neg B \lor (\Diamond A \land \Box \neg A)) \land \Box B
= \varphi'\\
$Tableaukalkuel:\\$
\begin{matrix}
        &s1, \varphi'\\
        &|\\
        &s1, \Box B\\
        &|\\
        &s1, \Box (\neg B \lor (\Diamond A \land \Box \neg A))\\
        &|\\
        &(s1, s2) \in E\\
        &|\\
        &s2, B\\
        &|\\
        &s2, (\neg B \lor (\Diamond A \land \Box \neg A))\\
        &l,r\\
        l:\\
        s2, \neg B(*)\\
        \lightning\\
        &&r:\\
        &&s2, (\Diamond A \land \Box \neg A)\\
        &&|\\
        &&(s2, s3) \in E\\
        &&|\\
        &&s3, \neg A\\
        &&|\\
        &&(s2, s4) \in E\\
        &&s4, \neg A\\
        &&s4, A\\
        &&\lightning\\
\end{matrix}\\
$
alternativ:\\
$       
        \varphi' 
\equiv  \Box (\neg B \lor (\Diamond A \land \neg \Diamond A)) \land \Box B\\
\equiv  \Box (\neg B \lor \bot) \land \Box B\\
\equiv  \Box \neg B \land \Box B\\
\equiv  \bot\\
$
\newpage
\subsection{Anwendung des Erfuellbarkeitstest}

\end{document}
