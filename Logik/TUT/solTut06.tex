\documentclass{article}
\author{Max Springenberg}
\title{Logik TUT 6}
\usepackage{amsmath}
\usepackage{amssymb}
\usepackage{stmaryrd}
\setcounter{section}{6}

\begin{document}
\maketitle
\newpage

\subsection{Kripkestrukturen}
\subsubsection\
\begin{tabular}{l||l|l|l|l|l|l|l}
        $s  \in K  $&1&2&2&3&3&4&4\\
        \hline
        $s^*\in K^*$&1&2&4&3&5&3&5\\
\end{tabular}\\
rausfallen werden:\\
$(1,1)$, da $(1,3) \in E$\\
$(3,3), (3,5)$, da $(3,1) \in E$\\
\\
damit ist die Menge aller Relationspaare:\\
$
S =\{(2,2), (2,4), (4,3), (4,5)\}\\
$
\subsubsection\
$
\varphi = \Diamond A \land \Diamond B\\
$
\\
wohlmoeglich auch minimalistischer loesbar mit:\\
$
\varphi' = \Diamond A\\
$
\newpage 
\subsection{Kripkestrukturen}
\subsubsection\
\begin{tabular}{l||l|l|l|l|l|l|l|l}
        $s_1 \in K_1$&1&1&2&3&4& 4&5&5\\
        \hline
        $s_2 \in K_2$&7&8&6&6&9&10&7&8\\
\end{tabular}\\
rausfallen werden:\\
$(1,8)$, da $(1,2) \in E_1$\\
$(2,6), (3,6)$, da $(6,7) \in E_2$\\
$(1,7)$, da $(7,6) \in E_2$\\
$(5,7)$, da $(5,4) \in E_1$\\
$(5,8)$, da $(8,7) \in E_2$\\
\\
damit ist die Menge alle Relationspaare:\\
$
S = \{(4,9), (4,10)\}
$
\subsubsection\
(i)\\
$
\varphi_i = \Box (B \land \Box (A \land B))\\
$
(ii)\\
nein, da $(4,9) \in S$\\
\subsubsection\
$
\begin{matrix}
        &A\\
        B   && B\\
        A,B && A,B\\
        A,B && A,B\\
\end{matrix}\\
$
\newpage
\subsection{Kripkestrukturen}
\subsubsection\
G := Zahl ist gerade\\
S := Zahl ist schwarz\\
\\
$
\varphi_1 = G \lor \Diamond G \lor \Diamond \Diamond G\\
\varphi_2 = S \rightarrow \Box \neg S\\
        \neg ((\neg G \land \neg S) \land \Diamond (\neg G \land \neg S))\\
\equiv  (\neg (\neg G \land \neg S) \lor \Box \neg (\neg G \land \neg S))\\
\equiv  (G \lor S \lor \Box (G \lor S))
= \varphi_3\\
\\
\varphi =   (G \lor \Diamond G \lor \Diamond \Diamond G) \land
            (S \rightarrow \Box \neg S) \land
            (G \lor S \lor \Box (G \lor S))\\
$
\subsubsection\
$
s1 = S \land G\\
s2 = \neg S\\
\\
(s1,s2) \in E\\
$
\end{document}
