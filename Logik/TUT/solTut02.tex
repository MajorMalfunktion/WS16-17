\documentclass{article}
\author{Max Springneberg}
\title{Logik TUT 2}
\usepackage{amsmath}
\usepackage{amssymb}
\usepackage{stmaryrd}
\setcounter{section}{2}

\begin{document}
\maketitle
\newpage

\subsection{Markierungsalgorithmus}
\subsubsection\
$
\varphi =
        (\neg A \lor (C \land \neg D \land E)) \land
        (F \rightarrow (\neg A \lor G)) \land
        ((F \land L) \rightarrow A) \land 
        L \land 
        \neg M \land
        (E \rightarrow (\neg C \lor \neg D)) \land
        F \land 
        G\\
\equiv  (\neg A \lor C) \land 
        (\neg A \lor \neg D) \land 
        (\neg A \lor E)) \land
        (\neg F \lor \neg A \lor G) \land
        ((F \land L) \rightarrow A) \land
        L \land
        \neg M \land
        (\neg E \lor \neg C \lor \neg D) \land
        F \land
        G\\
\equiv  (A \rightarrow C) \land 
        ((A \land D) \rightarrow \bot) \land 
        (A \rightarrow E)) \land
        ((F \land A) \rightarrow G) \land
        ((F \land L) \rightarrow A) \land
        (\top \rightarrow L) \land
        (M \rightarrow \bot) \land
        ((E \land C \land D) \rightarrow \bot) \land
        (\top \rightarrow F) \land
        (\top \rightarrow G)
= \varphi'\\
$
\\
$\varphi'$ ist offensichtlich Hornformel, 
damit kann der Markierungsalgorithmus durchgefuehrt werden.\\
\\
Markierte Literale:\\
Schritt 1:\\
F, G, L\\
\\
Schritt 2:\\
A\\
\\
Schritt 3:\\
C, E\\
\\
Schritt 4:\\
keine weiteren Schritte mehr moeglich, die Formel ist erfuellbar.\\
\subsubsection\
$
\psi =
        (\neg(D \land E) \lor \neg B) \land
        A \land
        (C \lor \neg A) \land
        (\neg A \lor ((E \lor \neg C) \land (\neg C \lor B)) \lor \neg D) \land 
        D\\
\equiv  ((D \land E \land B) \rightarrow \bot) \land
        (\top \rightarrow A) \land
        (A \rightarrow C) \land
        (
            \neg (A \land D) \lor
            ((E \lor \neg C) \land (\neg C \lor B))
        ) \land
        (\top \rightarrow D)\\
\equiv  ((D \land E \land B) \rightarrow \bot) \land
        (\top \rightarrow A) \land
        (A \rightarrow C) \land
        (
            (\neg (A \land D) \lor (E \lor \neg C)) \land 
            (\neg (A \land D) \lor (\neg C \lor B))
        ) \land
        (\top \rightarrow D)\\
\equiv  ((D \land E \land B) \rightarrow \bot) \land
        (\top \rightarrow A) \land
        (A \rightarrow C) \land
        (
            (\neg A \lor \neg D \lor E \lor \neg C) \land
            (\neg A \lor \neg D \lor \neg C \lor B)
        ) \land
        (\top \rightarrow D)\\
\equiv  ((D \land E \land B) \rightarrow \bot) \land
        (\top \rightarrow A) \land
        (A \rightarrow C) \land
        ((A \land D \land C) \rightarrow E) \land
        ((A \land D \land C) \rightarrow B) \land
        (\top \rightarrow D)
= \psi'\\
$
\\
$\psi'$ ist offensichtlich Hornformel, 
damit kann der Markierungsalgorithmus durchgefuehrt werden.\\
\\
Markierte Literale:\\
Schritt 1:\\
D, A\\
Schritt 2:\\
C\\
Schritt 3:\\
E, B\\
Schritt 4:\\
$(D \land E \land B) \rightarrow \bot) \lightning$\\
\\
damit ist $\psi'$ nicht erfuellbar\\
\newpage
\subsection{Aussagenlogische Resulotion}
\subsubsection\
$A$ aus $(C), (\neg C \lor A)$\\
$\neg C \lor \neg A$ aus $(B), (\neg B \lor \neg C \lor \neg A)$
$\neg A$ aus $(C), (\neg C \lor \neg A)$\\
$\Box$ aus $(A), (\neg A)$\\
\subsubsection\
$\neg B$ aus $(\neg D), (\neg B \lor D)$\\
$C$ aus $(E), (\neg E  \lor C)$\\
$A \lor B$ aus $(C), (A \lor B \lor \neg C)$\\
$A$ aus $(\neg B), (A \lor B)$\\
$F$ aus $(A), (\neg A \lor F)$\\
$\Box$ aus $(F), (\neg F)$\\
\end{document}
