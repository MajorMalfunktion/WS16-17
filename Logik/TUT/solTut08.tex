\documentclass{article}
\author{Max Springenberg}
\title{Logik TUT 8}
\usepackage{amsmath}
\usepackage{amssymb}
\usepackage{stmaryrd}
\setcounter{section}{8}

\begin{document}
\maketitle
\newpage

\subsection{Signaturen, Teilformeln, freie und gebundene Vriablen}
\subsubsection\
(i)\\
Signatur:\\
$\{R,S,g,f\}$\\
\\
atomare Teilformeln:\\
$R(x,y), S(x,y,z)$\\
\\
Grundterme:\\
$a, z$\\
\\
Terme:\\
Grundterme und $g(a), x, y, f(z)$\\
\\
freie Variablen:\\
a in $\exists x R(g(a),x)$, $\forall y (... \lor R(x,a))$\\
rest ist stets gebunden\\
\\
(ii)\\
Signatur:\\
$\{P,Q,g,f\}$\\
\\
atomare Teilformeln:\\
$P(x), Q(x,y)$\\
\\
Grundterme:\\
$f(x), y, z, x$\\
\\
Terme:\\
Grundterme und $g(f(x)), f(y), g(z), c$\\
\\
freie Variablen:\\
x in $P(x)$\\ 
x, y in $\exists z (... \land Q(g(f(x)), y))$\\
z, c in $\exists x \forall y (Q(y, g(z)) \lor ... \lor P(c))$\\
\subsubsection\
(i) x wird nicht benutzt\\
\\
(ii) R ist nur zweistellig\\
\\
(iii) $\checkmark$\\
\\
(iv) Relationssymbole koennen nicht verschachtelt werden.\\
\\
(v) c wird nicht in R benutzt\\
\\
(vi) keine Formel, aber ein Term\\
\newpage
\subsection{Bundesliga}
\subsubsection\
Die Beispiele zu den jeweiligen Mengen aus der Aufgabenstellung sollten genuegen.
Es waere meines Erachtens nicht sonderlich sinnvoll Spieltage und Mannschaften 
fuer die ganze Bundesliga-Saison rauszusuchen.\\
\subsubsection\
(Die alternativen Schreibweisen werden nicht benutzt, waeren aber intuitiver...)
$
(i)\ \forall t, m(S(t) \land M(m) \land (T(t,s,m) \rightarrow T(t,d,m))\\
$alternativ:\\$
\forall t \in S^A, m \in M^A (T(t,s,m) \rightarrow T(t,d,m))\\
\\
(ii)\ \forall m(M(m) \land T(l,d,m))\\
$alternativ:\\$
\forall m \in M^A (T(l,d,m))\\
\\
(iii)\ 
\forall m   (
                M(m) \land (T(l,s,m) \rightarrow (
                    T(e,s,m) \land \exists t (S(t) \land (t \neq e \land T(t,s,m)
                )
            )\\
$alternativ:\\$
\forall m \in M^A(T(l,s,m) \rightarrow (T(e,s,m) \land \exists t \in S^A (t \neq e \land T(t,s,m))))\\
$
\subsubsection\
(i) Dortmund und Schalke sind Mannschaften 
    und eine Mannschaft ist genau dann eine Mannschaft, wenn sie kein Spieltag ist (???)\\
(ii) Wenn Dortmund an jedem Spieltag vor Schalke ist, dann wird Schalke letzter.\\
(iii) Wenn eine Mannschaft an einem Spieltag vor allen anderen steht, 
      dann steht sie am kommendem Spieltag vor mindestens einer Mannschaft.\\
\end{document}
