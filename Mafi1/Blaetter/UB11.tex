\documentclass{article}
\title{MafI 1 UB07}
\usepackage{amsmath}
\usepackage{amssymb}
\usepackage{stmaryrd}
\usepackage[onehalfspacing]{setspace}

\onehalfspacing
\setcounter{section}{11}
\begin{document}
\maketitle
\newpage

\subsection{Linearkombinationen und lineare Abhaengigkeit}
\subsubsection\
$
\begin{pmatrix}
         3&-3& 3&|&1\\
         4& 4& 4&|&1\\
         5& 5&-5&|&1\\
\end{pmatrix}\\
\begin{pmatrix}
         3&-3& 3&|&1\\
         4& 4& 4&|&1\\
        30& 0& 0&|&8\\
\end{pmatrix}\\
\begin{pmatrix}
         3&-3& 3&|&1\\
         4& 4& 4&|&1\\
         1& 0& 0&|&\frac{4}{15}\\
\end{pmatrix}\\
\begin{pmatrix}
         0&-3& 3&|&1 - \frac{12}{15}\\
         0& 4& 4&|&1 - \frac{16}{15}\\
         1& 0& 0&|&\frac{4}{15}\\
\end{pmatrix}\\
\begin{pmatrix}
         0&-3& 3&|&1 - \frac{12}{15}\\
         0& 0&24&|&\frac{9}{15}\\
         1& 0& 0&|&\frac{4}{15}\\
\end{pmatrix}\\
\begin{pmatrix}
         0&-3& 3&|&1 - \frac{12}{15}\\
         0& 0& 1&|&\frac{1}{40}\\
         1& 0& 0&|&\frac{4}{15}\\
\end{pmatrix}\\
\begin{pmatrix}
         0&-3& 0&|&1 - \frac{12}{15} - \frac{3}{40}\\
         0& 0& 1&|&\frac{1}{40}\\
         1& 0& 0&|&\frac{4}{15}\\
\end{pmatrix}\\
\begin{pmatrix}
         0&-3& 0&|&1 - \frac{105}{120}\\
         0& 0& 1&|&\frac{1}{40}\\
         1& 0& 0&|&\frac{4}{15}\\
\end{pmatrix}\\
\begin{pmatrix}
         0& 1& 0&|&-\frac{15}{360}\\
         0& 0& 1&|&\frac{1}{40}\\
         1& 0& 0&|&\frac{4}{15}\\
\end{pmatrix}\\
\begin{pmatrix}
         0& 1& 0&|&-\frac{1}{24}\\
         0& 0& 1&|&\frac{1}{40}\\
         1& 0& 0&|&\frac{4}{15}\\
\end{pmatrix}\\
$
\\
damit ist $u = \frac{1}{40} * v_1 - \frac{1}{24} * v_2 + \frac{1}{40} * v_3$
\subsubsection\
$
    u - v
=   -(v - w + w - u)
\\
$damit dann auch$\\
    v - w
=   -(u - v + w - u)\\
    w - u
=   -(u - v + v - w)\\
$
\subsubsection\
$
\begin{pmatrix}
        1 & 1& 1& 1\\
        0 & 2&-3& 0\\
        1 & 2&-1& 0\\
        1 & 2& 2&-1\\
\end{pmatrix}\\
\begin{pmatrix}
        1 & 1& 1& 1\\
        0 & 2&-3& 0\\
        0 & 1&-2&-1\\
        0 & 1& 1&-2\\
\end{pmatrix}\\
\begin{pmatrix}
        1 & 1& 1& 1\\
        0 & 2&-3& 0\\
        0 & 0&-1&-1\\
        0 & 0& 5&-2\\
\end{pmatrix}\\
\begin{pmatrix}
        1 & 1& 1& 1\\
        0 & 2&-3& 0\\
        0 & 0&-1&-1\\
        0 & 0& 0&-7\\
\end{pmatrix}\\
\begin{pmatrix}
        1 & 0& 0& 0\\
        0 & 1& 0& 0\\
        0 & 0& 1& 0\\
        0 & 0& 0& 1\\
\end{pmatrix}\\
$
\\
damit linear unabhaengig.\\
\newpage
\subsection{Untervektorraum und Dimensionssatz}
\subsubsection\
$
((0,0,0,0) \in V_1) \Rightarrow V_1 \neq \emptyset\\
(x \in \mathbb{R}^4. x_1 = ... = x_4) \Rightarrow x \in V_1\\
(y \in \mathbb{R}^4. y_1 = ... = y_4) \Rightarrow y \in V_1\\
s \in \mathbb{R}\\
\\
x + y = (x_1 + y_1, ... , x_4 + y_4)^t \in V_1\\
s * x = (s * x_1, ... ,s * x_4)^t \in V_1\\
\\
$seien nun$\\
((0,0,0,0) \in V_2) \Rightarrow V_2 \neq \emptyset\\
(x \in \mathbb{R}^4. x_1 + ... + x_4 = 0) \Rightarrow x \in V_2\\
(y \in \mathbb{R}^4. y_1 + ... + y_4 = 0) \Rightarrow y \in V_2\\
s \in \mathbb{R}\\
\\
x + y = (x_1 + y_1, ... , x_4 + y_4)^t \in V_2\\
s * x = (s * x_1 + ... + s * x_4)^t \in V_2\\
$
\subsubsection\
$
B_{V_1} = {(1,1,1,1)^t}\\
dim V_1 = 1\\
\\
B_{V_2} = {(1,0,-1,0)^t,(0,1,0,-1)^t,(1,0,0,-1)^t,(0,1,-1,0)^t}\\
dim V_2 = 4\\
\\
B_{V_1 \cap V_2} = {(0,0,0,0)^t}\\
dim V_1 \cap V_2 = 1\\
\\
B_{V_1 + V_2} = {(1,1,1,1)^t, (1,0,-1,0)^t,(0,1,0,-1)^t,(1,0,0,-1)^t,(0,1,-1,0)^t}\\
dim V_1 + V_2 = 5\\
$
\newpage
\subsection{Basis eines EZS}
$
\begin{pmatrix}
        4 & 1& 1& 0&−2\\
        0 & 1& 4&-1& 2\\
        4 & 3& 9&-2& 2\\
        1 & 2& 3& 4& 5\\
        0 & 2& 8&-2& 4\\
\end{pmatrix}\\
\begin{pmatrix}
        4 & 1& 1& 0&−2\\
        0 & 1& 4&-1& 2\\
        4 & 3& 9&-2& 2\\
        1 & 2& 3& 4& 5\\
        0 & 0& 0& 0& 0\\
\end{pmatrix}\\
\begin{pmatrix}
        4 & 1& 1& 0&−2\\
        0 & 1& 4&-1& 2\\
        0 & 2& 8&-2& 0\\
        0 & 7&11&16&18\\
        0 & 0& 0& 0& 0\\
\end{pmatrix}\\
\begin{pmatrix}
        4 & 1& 1& 0&−2\\
        0 & 1& 4&-1& 2\\
        0 & 0& 0& 0&-4\\
        0 & 0&17& 9& 4\\
        0 & 0& 0& 0& 0\\
\end{pmatrix}\\
\begin{pmatrix}
        4 & 1& 1& 0& 0\\
        0 & 1& 4&-1& 0\\
        0 & 0&17& 9& 0\\
        0 & 0& 0& 0& 1\\
        0 & 0& 0& 0& 0\\
\end{pmatrix}\\
\begin{pmatrix}
        4 & 1& 1& 0& 0\\
        0 & 9&53& 0& 0\\
        0 & 0&17& 9& 0\\
        0 & 0& 0& 0& 1\\
        0 & 0& 0& 0& 0\\
\end{pmatrix}\\
$
\end{document}
