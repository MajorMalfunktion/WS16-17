\documentclass{article}
\author{Max Springenberg}
\title{MafI 1 UB02}
\usepackage{amssymb}
\usepackage{amsmath}
\setcounter{section}{2}

\begin{document}
\maketitle
\newpage

\subsection{Praedikatenlogik}
\subsubsection\
Praedikat kgV(n,m,x) ist dann erfuellt, wenn x das kgV von n und m ist\\
\\
$
kgV(n,m,x) =_{df} n|x \land m|x \land \forall y.(n|y \land m|y) \Rightarrow y \geq x\\
$
\subsubsection\
$
\forall n \in \mathbb{N}_{>2}. \exists p, p' \in P = \{2n - 1\}. 2n = p + p'\\
$
\subsubsection\
$
\neg(   
        \exists e \in \mathbb{R}. 
        e > 0 \land \forall n_0 \in \mathbb{N}.
        \exists n \in \mathbb{N}.
        n \geq n_0 \land \frac{1}{n} \geq e
    )\\
\equiv  \forall e \in \mathbb{R}.
        e > 0 \Rightarrow \exists n_o \in \mathbb{N}.
        \forall n \in \mathbb{N}.
        (\geq n_0) \Rightarrow (\frac{1}{n} < e)\\
$
\newpage
\subsection{Relation}
\subsubsection\
gesucht: $(R^{-1} \odot S) \odot T$\\
\\
$
R^{-1} = \{(4,2), (6,2) ,(6,3)\}\\
R^{-1} \odot S = \{(4,5), (4,6), (6,5), (6,6)\}\\
(R^{-1} \odot S) \odot T = \{(4,2), (4,6), (4,1), (4,5), (6,2), (6,6), (6,1), (6,5)\}\\
$
\subsubsection\
$
3|6 \land 3|9 \Rightarrow nicht\ rechtseindeutig\\
3|6 \land 2|6 \Rightarrow nicht\ linkseindeutig\\
1 \in \mathbb{N}
$
teilt alle Elemente von $\mathbb{N} \Rightarrow$ rechtstotal\\
Es ist linkstotalitaet (leider keine Loesung)\\
\subsubsection\
$
R = \emptyset = R^{-1}, A = \{1\}, I_A =\{(1,1)\} \\
R \odot R^{-1} \
$
\newpage
\subsection{Injektiv, Surjektiv}
\subsection\
$
sei\ F(x)=F(y)\\
                    (-1)^x * x = (-1)^y * y
\Leftrightarrow     x = y
\Rightarrow         injektiv\\
|\mathbb{N}| < |\mathbb{Z}| \Rightarrow
$
nach Schubfachprinzip nicht surjektiv\\
\subsubsection\
$
|\mathbb{Z}| > |\mathbb{N}| \Rightarrow
$
nach Schubfachprinzip nicht injektiv\\
$
\forall n \in \mathbb{N}.g(n) = n \Rightarrow surjektiv\\
$
\subsubsection\
injektiv, da Vorzeichenwechsel aus f durch g neutralisiert wird.\\
surjektiv, da\ 
$
\forall n \in \mathbb{N}. g \circ f (n)=|(-1)^n * n| = n 
$\\
\subsubsection\
nicht injektiv, da g nicht injektiv\\
nicht surjektiv, da f nicht surjektiv\\
\newpage
\subsubsection{Schubfachprinzip}
Beweis durch Wiederspruch:\\
Annahme: es ex. keine zwei nichtleere disj. Mengen $Y, Y' \subset X$\\
Summe Y = Summe Y'\\
\\
damit ex. fuer jede Teilmenge von X eine unterschiedliche Summe.\\
$2^{10} = 1024$ unterschiedliche Teilmengen.\\
\\
Die Summen sind $\in \{1,..,955\}$\\
\\
Das Schubfachprinzip sagt, dass es keine injektive Funkt. gibt, die den Teilmengen Summen zuordnet.\\
Somit muss es Teilmengen mit den selben Summen geben.
\end{document}
