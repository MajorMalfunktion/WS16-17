\documentclass{article}
\title{MafI 1 Merkzettel}
\usepackage{amsmath}
\usepackage{amssymb}
\usepackage{stmaryrd}
\usepackage[onehalfspacing]{setspace}

\onehalfspacing
\begin{document}
\maketitle
\section{Gruppen}
Eine Gruppe <Menge, Operant> gilt als solche, wenn sie:\\
\begin{tabular}{ll}
        Assoziativ          & $\rightarrow$ bzgl. Operant Assoziativ\\
        neutrl. Element     & $\rightarrow$ bzgl. Operant neutrl. Element\\
        inv. Element        & $\rightarrow$ bzgl. Operant inv. Element\\
\end{tabular}\\
ist.\\
\subsection{Gruppenhomorphismen}
Ein Gruppenhomorphismus ist Eine Funktionen h, fuer welche gilt:\\
$
h: <Menge,Operant> \rightarrow <Menge', Operant'>\\
a,b \in Menge\\
h(a\ `Operant`\ b) = h(a)\ `Operant`'\ h(b)\\
\Rightarrow h(a^{-1}) = h(a)^{-1}\\
$
\\
\section{Permutationen und Zyklen}
Eine Permutationen laesst sich in Zyklenshreibweise darstellen.\\
\\
Die Permutaion $\sigma$ laesst sich wie folgt darstellen:\\
normal:\\
$
\sigma = \begin{pmatrix}
        1&2&3&4&5&6\\
        2&3&4&5&1&6\\
\end{pmatrix}
\begin{matrix}
        p_n\\
        \downarrow\\
        p_{n+1}\\
\end{matrix}\\
$
\\
Zyklenshreibweise:\\
$
\sigma = (12345)(6)\\
$
\\
Rechenregel:\\
$
    \sigma_n \circ \sigma_{n+1} 
=   \begin{pmatrix}
        t & t'  & ...\\
        b & d   & ...\\
\end{pmatrix} \circ \begin{pmatrix}
        c & a   & ...\\
        t' & t  & ...\\
\end{pmatrix}\\
=   \begin{pmatrix}
        a & c   & ...\\
        b & d   & ...\\
\end{pmatrix} = (abcd\ ...)\\
$
Vergleichbar mit Transitivitaet.\\
\section{Verband}
<V,Operator> ist dann ein Verband, wenn:\\
\begin{tabular}{ll}
        Reflexivitaet   &\\
        Antisymetrie    &\\
        Transitivitaet  &\\
\end{tabular}\\
gilt.\\
\\
\subsection{Vollstaendiger Verband}
ein Verband ist vollstaendig, wenn:\\
hinsichtlich des Infimums ein neutr. Element, inv. Element und Abgeschlossenhiet existiert.\\
\end{document}
