\documentclass{article}
\title{MafI 1 UB06}
\usepackage{amsmath}
\usepackage{amssymb}
\usepackage{stmaryrd}
\setcounter{section}{6}

\begin{document}
\maketitle
\newpage

\subsection{Distributive Verbaende}
\subsubsection\
$
    d \curlywedge (a \curlyvee c)
=   d \curlywedge 1
=   d\\
    (d \curlywedge a) \curlyvee (d \curlywedge c)
=   a \curlyvee 0 
=   a\\
\Rightarrow$ nicht distributiv\\
\subsubsection\
Tupel der komp. Elemente: 
$\{ (1,0), (a,e), (d,c), (a, c)\}$\\
\\
$
d \curlywedge e = b, b \notin S\\
\Rightarrow$ kein Unterverband\\
\subsubsection\
Menge der Elemente mit komplementaeren Elementen von $V$: $S$\\
$
z.z.:\\
\forall x,y \in S. (x \curlywedge y) \in S \land (x \curlyvee y) \in S\\
\\
x, y, \tilde x, \tilde y \in S\\
\tilde x, \tilde y$ sind disjunkte Komplementaere von $x, y\\
\\
    (x \curlywedge y) \curlywedge (\tilde x \curlyvee \tilde y)
=   (x \curlywedge y \curlywedge \tilde x) \curlyvee (x \curlywedge y \curlywedge \tilde y)\\
=   0 \curlyvee 0 
=   0\\
\\
    (x \curlywedge y) \curlyvee (\tilde x \curlyvee \tilde y)
=   (x \curlyvee y \curlyvee \tilde x) \curlywedge (x \curlyvee y \curlyvee \tilde y)\\
=   1 \curlywedge 1 
=   1\\
$
Damit hat $x \curlywedge y$ das komplementaere Element $(\tilde x \curlyvee \tilde y)$,
sowie $x \curlyvee y$ $(\tilde x \curlyvee \tilde y)$.\\
Damit sind diese Elemente auch in S.\\
\newpage
\subsection{Boolscher Verband}
\subsubsection\
\begin{tabular}{ccccc}
        &&30\\
        6&&15&&10\\
        &3&2&5\\
        &&1\\
\end{tabular}\\
$
(3,2),(2,5),(3,5) \rightarrow^{\curlywedge} (1)\\
$
damit dann auch $1_B = 30, 0_B = 1$\\
\subsubsection\
z.z.:\\
A abgeschlossen unter $(\curlywedge, \curlyvee)$\\
\\
(hatte keine Lust $\forall x, y \in A. (x \curlyvee y), \forall x, y \in A. (x \curlywedge y)$)
zu zeigen\\
Mit einer Art Lambda Ausdruck ueber Tupel fuer $(\curlywedge, \curlyvee)$:\\
$
(6,15),(15,10),(6,10) \rightarrow^{\curlyvee} (30),\\
(3,2) \rightarrow^{\curlyvee} (6),\\
(2,5) \rightarrow^{\curlyvee} (10),\\
(3,5) \rightarrow^{\curlyvee} (15),\\
etc\ ...\\
$
Es folgt, dass A bzgl. $(\curlywedge, \curlyvee)$ abgeschlossen ist.\\
\\
z.z.:\\
A ist distributiv\\
\\
es gilt: $A \subseteq B \land A$ abgeschlossen\\
damit ist A dann auch wie B distributiv\\
\\
z.z.:\\
$
\exists 0_A, 1_A \in A. 
\forall x \in A \exists \bar x \in A. 
x \curlywedge \bar x = 0_A \land x \curlyvee \bar x = 1_A\\
\\
$
Diese existieren mit
$
0_A = 3, 1_A = 30
\\
\bar 30  \in (B, \curlywedge, \curlyvee)\\
\\
    (30 \curlywedge \bar 30) \curlyvee 3 
=   (1 \curlyvee 30)
=   1 \curlyvee 3
=   3\\
$
\newpage
\subsection{Vollstaendiger Verband}
\subsubsection\
(Assoziativitaet, Kommutativitaet, Absorption sind offensichtlich ...)\\
Vollstaendigkeit:\\
\\
sei $X \subseteq A \times X$ eine Menge mit:\\
$
X_A = \{a \in A | \exists b \in B. (a, b) \in X\}\\
X_B = \{b \in B | \exists a \in A. (a, b) \in X\}\\
$
so gilt fuer $(a,b) \in X$ bel.:\\
$
(inf(X_A), inf(X_B)) \preceq (a,b)
$, da \\$
inf(X_A, a) = inf(X_A)\\
inf(X_B, b) = inf(X_B)\\
$
analog fuer sup...
\newpage
\subsubsection{Verband der Funktionen}
hier gabs keine Mitschriften
\end{document}
