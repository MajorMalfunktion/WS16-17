\documentclass{article}
\title{MafI 1 UB04}
\usepackage{amsmath}
\usepackage{amssymb}
\usepackage{stmaryrd}
\setcounter{section}{4}

\begin{document}
\maketitle
\newpage
\subsection{Induktives Definieren}
\subsubsection\
$
f(0) = k^0\\
f(n) = f(n-1) * k\\
$
\subsubsection\
Eine n-elementige Menge hat $2^n$ Teilmengen\\
$
f(0) = 2^0 = 1\\
f(n) = f(n-1) * 2\\
$
\newpage
\subsection{Backus-Natur-Form}
\subsubsection\
$
<AT> ::= <Dezimalzahl> | <V> | (-<AT>) | (<AT> + <AT>) | (<AT> * <AT>)\\
<V> ::= X | Y | Z\\
\\
            <AT>
\Rightarrow (<AT> * <AT>)\\
\Rightarrow ((<AT> * <AT>) * (-<AT>))\\
\Rightarrow (((<AT> * <AT>) * (<AT> + <AT>)) * (-(-<AT>)))\\
\Rightarrow ((((-<AT>) * <AT>) * (<AT> + <AT>)) * (-(-<AT>)))\\
\Rightarrow ((((-<V>) * <Dezimalzahl>) * (<V> + <V>)) * (-(-<Dezimalzahl>)))\\
\Rightarrow ((((-X) * 5) * (X + Y)) * (-(-3)))\\
$
\subsubsection\
$
<Dezimalzahl> ::= <Zahl><Dezimalzahl> | <Zif>\\
<Zahl> ::= 1 | .. | 9\\
<Zif> ::= 0 | .. | 9\\
$
\newpage
\subsection{Partielle Ordnung, Quasiordnung}
\subsubsection\
$
reflexiv:\\
\{(a,a), (b,b), (c,c)\} \in R\\
offensichtlich\ transitiv\\
antisymetrisch:\\
offensichtlich\ (a = a, b = b, c = c)\\
\Rightarrow R\ ist\ partielle\ Ordnung\\
$
\subsubsection\
$
reflexiv:\\
|x| \leq |x| \Rightarrow x R x\\
transitiv:\\
|x| \leq |y| \land |y| \leq |z| \Rightarrow |x| \leq |z| \Rightarrow x R z\\
antisymetrisch:\\
|-1| \leq |1| \land |1| \leq |-1|\\
-1 \neq 1 \lightning\\
\Rightarrow R\ ist\ Quasi-Ordnung\\
$
\subsubsection\
$
reflexiv:\\
x = y = x \Rightarrow x R x\\
transitiv:\\
x = y \land y = z  \Rightarrow x = z\\
|x| < |y| \land |y| < |z| \Rightarrow |x| < |z| \Rightarrow x R z\\
antisymetrisch:\\
(|x| < |y| \lor x = y) \land (|y| < |x| \lor y = x) \Rightarrow x = y\\
\Rightarrow R\ ist\ partielle\ Ordnung\\
$
\newpage
\subsection{Induktives Definieren}
\subsubsection\
Anzahl der moeglichen Partitionen $\pi (n,k)$, der n-Elementigen Menge $M$.\\
$
\pi (n,k) = \begin{cases}
                1,                              & k = 1 \lor k = n\\
                \pi (n-1,k-1) + k * \pi(n-1,k), & sonst\\
            \end{cases}
$
\subsubsection\
$
    P(5)
=   \pi(5,5) + \pi(5,4) + \pi(5,3) + \pi(5,2) + \pi(5,1)\\
=   2 + \pi(5,4) + \pi(5,3) + \pi(5,2)\\
=   52\\
\\
N.R\ :\\
\pi(5,4) = \pi(4,3) + 4 * \pi(4,4) = (\pi(3,2) + 3) + 4 = \pi(2,1) + 9 = 10\\
\pi(5,3) = \pi(4,2) + 3 * \pi(4,3) + \pi(3,1) + 2 * \pi(3,2) + 18 = 25\\
\pi(5,2) = 15\\
$
\end{document}
