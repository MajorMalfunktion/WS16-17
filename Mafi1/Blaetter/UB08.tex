\documentclass{article}
\author{Max Springenberg}
\title{UB 08}
\usepackage{ngerman}
\usepackage{amsfonts}
\usepackage{stmaryrd}
\usepackage[table]{xcolor}
\usepackage{hhline}
\setcounter{section}{8}

\begin{document}
\maketitle
\newpage
%   A8.1
\subsection 
\\
%   A8.1.1
\subsubsection
\\
%   Begriffe
Gruppe: Monoid = inverses Element\\
Monoid: neutrales Element, Halbgruppe\\
Halbgruppe: Assoziativit"at\\
endliche Gruppe: endliche Anzahl der Elemente, neutrales Element, 
inversees Element, Assoziativit"at\\ 
\\
%   Rechnung 
$h: <G,\oplus> \times <G,\oplus> \rightarrow <G,\oplus>'x\\
\forall a, b, c \in G. h(a) = a \oplus x =h(b) =  b \oplus x\\
\Leftrightarrow h(a) = h(b)\\
\Leftrightarrow a \oplus x = b \oplus x
\Leftrightarrow a = b$\\

%   A8.1.2
\subsubsection \\ 
{
\rowcolors{1}{white}{gray!20}
\begin{tabular}{|l||l|l|l|l|l|l|l|}
        \hline
        $\oplus$&a  &b&c&d&e&f\\
        \hhline{|=||=|=|=|=|=|=|}
        a       &f  &d  &e  &b  &c  &a\\
        \hline
        b       &e  &f  &d  &c  &a  &b\\
        \hline
        c       &d  &e  &f  &a  &b  &c\\
        \hline
        d       &c  &a  &b  &e  &f  &d\\
        \hline
        e       &b  &c  &a  &f  &d  &e\\
        \hline
        f       &a  &b  &c  &d  &e  &f\\
        \hline
\end{tabular}
}
\newpage

%   A   8.2
\subsection
\\
%   A8.2.1
\subsubsection \\
$T = \{ \lambda_a | a \in G \}\\
\lambda_a (x) = a \oplus x\\
<T, \circ >\\
\forall a, b, c \in T\\
\\
%   1. Kriterium
Assoziativit"at:\\
\lambda_a \circ (\lambda_a \circ \lambda_c (x))\\
= \lambda_a \circ (\lambda_b \circ (\lambda_c (x)))\\
= \lambda_a (\lambda_b(\lambda_c (x)))\\
= \lambda_a(\lambda_b(c \oplus x))\\
= \lambda_a (b \oplus c \oplus)\\
= (a \oplus b \oplus c \oplus x)\\
= (a \oplus b) \circ \lambda_c (x)\\
= (\lambda_a (b \oplus e)) \circ \lambda_c (x)\\
= (\lambda_a \circ \lambda_b) \circ \lambda_c (x)$\\
\\
%   2. Kriterium
neutrales Element:\\
$\lambda_a \circ \lambda_e (x)\\
= a \oplus e \oplus x\\
= a \oplus x\\
= e \oplus a \oplus x\\
\lambda_e \circ \lambda_a (x)$\\
\\
%   3. Kriterium
inverses Element:\\
$\lambda_a \circ \lambda_{a^{-1}} (x) = \lambda_e (x)\\
a \oplus a^{-1} \oplus x = e \oplus x$\\
\\
%   A8.2.2
\subsubsection\\
$h : <G, \oplus> \rightarrow <T, \circ>\\
h(a)=\lambda_a\\
h(a \oplus b) =^! h(a) \circ h(b)\\
h(a \oplus b) = \lambda_{a \oplus b}(x)\\
= a \oplus b \oplus x\\
= \lambda_a (b \oplus x)\\
= \lambda_a(\lambda_b(x))\\
= \lambda_a \circ \lambda_b (x)\\
= h(a) \circ h(b)$
\newpage

%   A8.3
\subsection\\
%   A8.3.1
\subsubsection\\
$<A, \oplus, \odot> B \subseteq A\\
<\{a \in A | b \odot a = a \odot b, \forall b \in B\}, \oplus, \odot>$
ist Unterring von $<A, \oplus, \odot>$\\
\\
Tipp:\\
$( \forall a, b \in A : a \odot (-b) 
= (-a) \odot b 
= -(a \odot b) \land (-a) \odot (-b) 
= a \odot b)\\$
\\
Kriterien Bez"uglich des Plusoperanden:\\
Assoziativit"at:\\
Aus $A' \subseteq A$ folgt, dass 
$\forall a, c, d \in A'. a \oplus (c \oplus d) 
= (a \oplus c) \oplus d$ gilt.\\
\\
neutrales Element:\\
$b \odot  0 = 0 \odot b \Rightarrow 0 in jedem A'$ (Folie 272)\\
\\
inverses Element:\\
$b \odot (-a) = -(b \odot a) = -(a \odot b) = (-a) \odot b$\\
\\
Kommutativit"at:\\
$\forall a, c \in A'. a\oplus c = c \oplus a$, da $A' \subseteq A$\\
\\
Kriterien Bez"uglich des Maloperanden:\\
Assoziativit"at:\\
$\forall a, c, d \in A. a \odot (c \odot d) = (a \odot c) \odot d$, da $A' \subseteq A$\\
\\
%   A8.3.2
\subsubsection\\
Ring mit 0 = 1 sei ein Nullring!\\
Annahme $a \in R \land a \neq 0\\
a \odot 1 = a\ \lightning\\
0 = 1\ m"usste\ a \odot 1 = 0$
\end{document}
