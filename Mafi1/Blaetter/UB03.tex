\documentclass{article}
\title{MafI 1 UB03}
\setcounter{section}{3}
\usepackage{amsmath}
\usepackage{amssymb}

\begin{document}
\maketitle
\newpage
\subsection{Aequivalenzrealtionen}
\subsubsection\
z.z.: genau dann euklidisch, wenn Aequivalenzrelation\\
\\
"$\Rightarrow$":\\
Reflexivitaet ist bereits gegeben\\
\\
Symmetrie:\\
$
\forall a_1, a_2 \in A. a_1 R a_2 \Rightarrow^{!} a_2 R a_1
a_1 R a_1(aus\ Reflexivitaet)\\
\\
\forall a_1, a _2 \in A(a_1 R a_2 \land a_1 R a_1) \Rightarrow a_2 R a_1\\
damit\ a_1 R a_2 \land a_1 R a_1 \Rightarrow a_2 R a_1\\
$
\\
Transitivitaet:\\
$
\forall a_1, a_2, a_3 \in A. a_1, a_2, a_3 \in A. a_1 R a_2 \land a_2 R a_3 \Rightarrow^{!} a_1 R a_3\\
a_2 R a_1(aus\ Symmetrie)\\
\forall a_1, a_2, a_3 \in A.(a_2 R a_1 \land a_2 R a_3) \Rightarrow a_1 R a_3\\
damit\ a_1 R a_2 \land a_2 R a_3 \Rightarrow a_1 R a_3\\
$
\\
"$\Leftarrow$":\\
$
\forall a, b, c \in A.(a R b \land a R c) \Rightarrow^{!} b R c\\
\\
b R a (aus\ Symmetrie)\\
aus\ Transitivitaet:\\
b R a \land a R c \Rightarrow b R c\\
damit\ a R b \land a R c \Rightarrow b R c\\
$
\subsubsection\
R ist keine Aequivalenzrealtion, da die Transitivitaet nicht gilt\\
\\
$
\{1,2\} \cap \{2,3\} \neq \emptyset\\
\{2,3\} \cap \{3,4\} \neq \emptyset\\
aber\ \{1,2\} \cap \{3,4\} = \emptyset\\
$
\newpage
\subsection{Aequivalenzrealtionen und Partitionen}
\subsubsection\
$
\backsim_R =   
        \{\\
            (e,e), (f,d), (c,a), (b,f),\\
            (a,a), (b,b), (c,c), (d,d),(f,f),   \leftarrow \ Reflexivitaet\\
            (d,f), (a,c), (f,b),                \leftarrow \ Symmetrie\\
            (b,d), (d,b)                       \leftarrow \ Transitivitaet\ \&\ Symmetrie\\
        \}\\
$
\subsubsection\
$M \setminus \backsim_R$ ist Partition  basieredn auf Aequivalenzklassen\\
$
    M \setminus \backsim_R 
=   \{\{a, c\}, \{b, d, f\}, \{e\}\}\\
=   \{[a]_{\backsim_R},[b]_{\backsim_R},[e]_{\backsim_R}\}\\
$
\subsubsection\
$
g(x) =  \begin{cases}
            1, & x = a \lor x =c\\
            2, & x = b \lor x = d \lor x = f\\
            3, & x = e\\
        \end{cases}\\
$
\newpage
\subsection{Maechtigkeit von Mengen}
sei $P_e(\mathbb{N})$ die Menge aller unendlichen Teilmengen von $\mathbb{N}$.\\
z.z.: $P_e(\mathbb{N})$ abzaehlbar unendlich $\Leftrightarrow P_e(\mathbb{N}) \cong \mathbb{N}$\\
\\
Es reicht, den Satz von Cantor, Bernstein und Schroeder nachzuweisen.\\
$
P_e(\mathbb{N}) \leqq^! \mathbb{N}, \mathbb{N} \leqq^! P_e(\mathbb{N})\\
\\
P_e(\mathbb{N}) \leqq \mathbb{N}:\\
f : \mathbb{N} \rightarrow P_e(\mathbb{N})\\
f(n) := \{n\},\ offensichtlich\ injektiv\\
\\
\mathbb{N} \leqq  P_e(\mathbb{N}):\\
g: P_e(\mathbb{N}) \rightarrow \mathbb{N}\\
g(X) := \sum_{n \in X}{2^n},\ nicht\ surjektiv\\
$
\end{document}
