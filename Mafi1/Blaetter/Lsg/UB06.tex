\documentclass{article}
\title{MafI 1 UB06}
\usepackage{amsmath}
\usepackage{amssymb}
\usepackage{stmaryrd}
\setcounter{section}{6}

\begin{document}
\maketitle
\newpage

\subsection{Distributive Verbaende}
\subsubsection\
$
    d \curlywedge (a \curlyvee c)
=   d \curlywedge 1
=   d\\
    (d \curlywedge a) \curlyvee (d \curlywedge c)
=   a \curlyvee 0 
=   a\\
\Rightarrow$ nicht distributiv\\
\subsubsection\
Tupel der komp. Elemente: 
$\{ (1,0), (a,e), (d,c), (a, c)\}$\\
\\
$
d \curlywedge e = b, b \notin S\\
\Rightarrow$ kein Unterverband\\
\subsubsection\
Menge der Elemente mit komplementaeren Elementen von $V$: $S$\\
$
z.z.:\\
\forall x,y \in S. (x \curlywedge y) \in S \land (x \curlyvee y) \in S\\
\\
x, y, \tilde x, \tilde y \in S\\
\tilde x, \tilde y$ sind disjunkte Komplementaere von $x, y\\
\\
    (x \curlywedge y) \curlywedge (\tilde x \curlyvee \tilde y)
=   (x \curlywedge y \curlywedge \tilde x) \curlyvee (x \curlywedge y \curlywedge \tilde y)\\
=   0 \curlyvee 0 
=   0\\
\\
    (x \curlywedge y) \curlyvee (\tilde x \curlyvee \tilde y)
=   (x \curlyvee y \curlyvee \tilde x) \curlywedge (x \curlyvee y \curlyvee \tilde y)\\
=   1 \curlywedge 1 
=   1\\
$
Damit hat $x \curlywedge y$ das komplementaere Element $(\tilde x \curlyvee \tilde y)$,
sowie $x \curlyvee y$ $(\tilde x \curlyvee \tilde y)$.\\
Damit sind diese Elemente auch in S.\\
\newpage
\subsection{Boolscher Verband}
\subsubsection\
\begin{tabular}{ccccc}
        &&30\\
        6&&15&&10\\
        &3&2&5\\
        &&1\\
\end{tabular}\\
\\
Die komplementaeren Elemente in Tupeln sind:\\
$\{
        (5,6), (3,10), (2,15), (1,30)
\}$\\
\\
mit $\{(x_{n_0},y_{n_0}), ...\} \rightarrow^{Relationsoperator} \{x_{n_0}\ `Relationsoperator`\ y_{n_0}, ...\}$:\\
$
 \{(5,6), (3,10), (2,15), (1,30)\} \rightarrow^{\curlywedge} \{1\}\\
 \{(5,6), (3,10), (2,15), (1,30)\} \rightarrow^{\curlyvee} \{30\}\\
$
damit dann auch $1_B = 30, 0_B = 1$\\
\subsubsection\
z.z.:\\
A abgeschlossen unter $(\curlywedge, \curlyvee)$\\
\\
z.z.:\\
$\forall x, y \in A. \exists (x \curlyvee y), \forall x, y \in A. \exists (x \curlywedge y))\\
\\
\\
    \{(x_{n_0},y_{n_0}), ...\} \rightarrow^{Relationsoperator}\\
=   \{(y_{n_0},x_{n_0}), ...\} \rightarrow^{Relationsoperator} \{x_{n_0}\ `Relationsoperator`\ y_{n_0}, ...\}\\
a \in A\\
\\
\{(3,a)\} \rightarrow^{\curlyvee} \{(a)\},\\
\{(30,a)\} \rightarrow^{\curlyvee} \{(30)\},\\
\{(6,(a \neq 3)),(15,(a \neq 3))\} \rightarrow^{\curlyvee} \{(30)\},\\
\\
\{(3,a)\} \rightarrow^{\curlywedge} \{(3)\},\\
\{(30,a)\} \rightarrow^{\curlywedge} \{(a)\},\\
\{(6,(a \neq 30)),(15,(a \neq 30))\} \rightarrow^{\curlywedge} \{(3)\},\\
$
\\
Es folgt, dass A bzgl. $(\curlywedge, \curlyvee)$ abgeschlossen ist.\\
\\
z.z.:\\
A ist distributiv\\
\\
es gilt: $(A \subseteq B) \land A$ abgeschlossen\\
damit ist A dann auch wie B distributiv\\
\\
z.z.:\\
$
\exists 0_A, 1_A \in A. 
    \forall x \in A \exists \bar x \in A. 
        x \curlywedge \bar x = 0_A \land x \curlyvee \bar x = 1_A\\
\\
$
Diese existieren zu der Menge aller Komplementaeren Elemente:\\
$S \subseteq A, S = \{(3,30), (6,15)\}\\$
\\
wie bereits gezeigt:\\
$
\{(3,a)\} \rightarrow^{\curlywedge} \{(3)\},\\
\{(6,(a \neq 30)),(15,(a \neq 30))\} \rightarrow^{\curlywedge} \{(3)\},\\
\{(30,a)\} \rightarrow^{\curlyvee} \{(30)\},\\
\{(6,(a \neq 3)),(15,(a \neq 3))\} \rightarrow^{\curlyvee} \{(30)\},\\
$
\\
daraus folgt:\\
$0_A = 3, 1_A = 30$
\newpage
\subsection{Vollstaendiger Verband}
\subsubsection\
Assoziativitaet, Kommutativitaet, Absorption sind offensichtlich\\ 
\\
\\
Vollstaendigkeit:\\
\\
sei $X \subseteq A \times X$ eine Menge mit:\\
$
X_A = \{a \in A | \exists b \in B. (a, b) \in X\}\\
X_B = \{b \in B | \exists a \in A. (a, b) \in X\}\\
$
\\
so gilt fuer $(a,b) \in X$ bel.:\\
$(inf(X_A), inf(X_B)) \preceq (a,b)$, da \\
$
inf(X_A, a) = inf(X_A)\\
inf(X_B, b) = inf(X_B)\\
$
analog fuer sup...
\newpage
\subsection{Verband der Funktionen}
hier gabs keine Mitschriften
\end{document}
