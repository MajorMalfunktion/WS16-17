\documentclass{article}
\title{MafI 1 UB07}
\usepackage{amsmath}
\usepackage{amssymb}
\usepackage{stmaryrd}
\setcounter{section}{7}
\begin{document}
\maketitle
\newpage

\subsection{Strukturtafeln und Gruppen}
\subsubsection\
\begin{tabular}{ccccccc}
        $+_6$   &0&1&2&3&4&5\\
        \hline
        0       &0&1&2&3&4&5\\
        1       &1&2&3&4&5&0\\
        2       &2&3&4&5&0&1\\
        3       &3&4&5&0&1&2\\
        4       &4&5&0&1&2&3\\
        5       &5&0&1&2&3&4\\
\end{tabular}\\
\\
\begin{tabular}{ccccccc}
        $*_6$   &0&1&2&3&4&5\\
        \hline
        0       &0&0&0&0&0&0\\
        1       &0&1&2&3&4&5\\
        2       &0&2&4&0&2&4\\
        3       &0&3&0&3&0&3\\
        4       &0&4&2&0&4&2\\
        5       &0&5&4&3&2&1\\
\end{tabular}\\
\subsubsection\
z.z.:\\
$(\mathbb{Z}_6, +_6)$ soll Gruppe sein\\
\\
Assoziativitaet\\
offensichtlich, da normale Addition assoziativ ist.\\
\\
neutr. Element:\\
offensichtlich 0, da 0 das neutr. Element der normalen Addition ist.\\
\\
inv. Element:\\
$
\forall x \in \mathbb{Z}_6 \exists x' \in \mathbb{Z}_6.
x +_6 x' = 0 \Rightarrow x' = 6 - x\\
$
inv. Element ex. auch, damit ist $(\mathbb{Z}_6, +_6)$ eine Gruppe\\
\newpage
\subsection{Permutationen, Zyklenschreibweise}
\subsubsection\
$
\sigma_1 = (123456)\\
\sigma_2 = (13652)(4)\\
\sigma_3 = (126)(354)\\
$
\subsubsection\
$
    \sigma_1 \circ \sigma_2 \circ \sigma_3 
=   \begin{pmatrix}
            1&2&3&4&5&6\\
            2&6&3&1&5&4\\
    \end{pmatrix}
=   (1264)(3)(5)\\
$
\newpage
\subsection{Gruppe und Gruppenhomorphismus}
\subsubsection\
$
    (a \oplus b)^{-1}
=   e \oplus (a \oplus b)^{-1}
=   e \oplus b \oplus b^{-1} \oplus (a \oplus b)^{-1}
=   a \oplus a^{-1} \oplus b \oplus b^{-1} \oplus (a \oplus b)^{-1}
=   a^{-1} \oplus b^{-1}\\
$
\subsubsection\
"$\Rightarrow$" h ist injektiv:\\
$
\Rightarrow $jedes Element hat max. ein Urbild$\\
\Rightarrow Kern(h)$ist max. einelementig$\\
    h(e) \oplus' h(e)
=   h(e \oplus e)
=   h(e)
=   h(e) \oplus e'\\
    h(e)
=   e'\\
$
\\
"$\Leftarrow$" $Kern(h) = \{e\} \Rightarrow$ h ist injektiv:\\
Kontrraposition:\\
h nicht injektiv $\Rightarrow Kern(h) \neq \{e\}$\\
z.z.:\\
$
\exists a \in G.
h(a) = e'\\
$somit $a \in Kern(h) \land a \neq e\\
$
\\
sei h nicht injektiv:\\
$
\exists x, x' \in G. 
x \neq x'.
h(x) = h(x')\\
\\
$So ist:$\\
    e' 
=   h(x) \oplus' h(x)^{-1}
=   h(x) \oplus' h(x')^{-1}
=   h(x) \oplus' h(x'^{-1})
=   h(x \oplus x'^{-1})
\neq e' \lightning\\
$
\newpage
\subsection{Untergruppenverband}
$
V = \{V | V \subseteq G, V \cup G\}\\
\leq \subseteq V \times V\\
\\
                A \leq B 
\Leftrightarrow A \cup G$ von $B
\Leftrightarrow A \subseteq B\\
\\
(i)\\
z.z.:\\
<V,\leq>$ ist Verband$\\
\\
$partielle Ordnung:\\
ist reflexiv, da $A \leq A$ durch $\forall A \in V. A = A$ gilt\\
\\
antisymmetrisch:\\
$
A, B \in V.A \leq B \land B \leq A\\
\Rightarrow A \subseteq B \land B \subseteq A\\
\Rightarrow A = B\\
$
\\
transitiv:\\
$
A, B ,C \in V.A \leq B \land B \leq C\\
\Rightarrow A \subseteq B \land B \subseteq C\\
\Rightarrow A \subseteq C
\Rightarrow A \leq C\\
$
\\
\\
(ii)\\
z.z.:\\
$<V,\leq>$ ist vollstaendiger Verband$\\
\\
U \subseteq V\\
\\
inf(U) = \bigcap_{X \in U} X\\
\forall X \in U. \bigcap_{X \in U} X \subseteq X$ damit untere Schranke$\\
\\
\forall X \subseteq U \exists K. K \subseteq X\\
\Rightarrow K \subseteq \bigcap_{X \in U} X\\
\\
$Damit $inf(U) \in V$ muss $inf(U) \cup G$ von G sein.$\\
$
\\
abgeshlossen:\\
$
a, b \in inf(U)\\
\Rightarrow a, b \in \bigcap_{X \in U} X\\
\Rightarrow a, b \in X \forall X \in U
\Rightarrow a \oplus b \in X \forall X \in U\\
\Rightarrow a, b \in inf(U)\\
$
\\
neutr. Element:\\
$
e \in X \forall X \in U
\Rightarrow e \in inf(U)
$
\\
inv. Element:\\
sei
$
a \in inf(U)
\Rightarrow a \in X \forall X \in U
\Rightarrow a^{-1} \in X \forall X \in U\\
\Rightarrow a^{-1} \in inf(U)\\
$
\\
damit ist $<V, \leq>$ ein vollstaendiger Verband
\end{document}
