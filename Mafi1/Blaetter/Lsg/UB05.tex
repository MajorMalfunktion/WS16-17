\documentclass{article}
\usepackage{amsmath}
\usepackage{amssymb}
\usepackage{stmaryrd}
\title{MafI 1 UB05}
\setcounter{section}{5}

\begin{document}
\maketitle
\newpage

\subsection{induktives Definieren, strukturelle Induktion}
\subsubsection\
$
tm(\top) = \{\top\}\\
tm(\bot) = \{\bot\}\\
tm(V) = \{V\}\\
tm(\neg t_1) = \{\neg t1\} \cup tm(t_1)\\
tm(t_1 \land t_2) = \{t_1 \land t_2\} \cup tm(t_1) \cup tm(t_2)\\
tm(t_1 \lor t_2) = \{t_1 \land t_2\} \cup tm(t_1) \cup tm(t_2)\\
$
\subsubsection\
I.A.:\\
$
t \in \{\top, \bot\} \cup V \subseteq BT\\
|tm(t)| = |\{t\}| = 1 \leq 1 = l(t)\\
$
\\
I.V.:\\
Die Aussage gelte fuer bel., aber feste $t_1, t_2 \in BT$.\\
\\
I.S.:\\
$
(i)\\
    |tm(\neg t_1)| 
=   |\{\neg t_1\} \cup tm(t_1)|
=   |\{\neg t_1, t_1\}| 
=   2 \leq 2 
=   1 + 1
=   1 + l(t_1)\\
\\
(ii)\\
    |tm(t_1 \lor t_2)|
=   |\{t_1 \lor t_2\} \cup tm(t_1) \cup tm(t_2)|
=   |\{t_1 \lor t_2, t_1, t_2\}|
=   3 \leq 3
=   1 + 1 + 1
=   1 + l(t_1) + l(t_2)
=   l(t_1 \lor t_2)\\
\\
(iii)\\
    |tm(t_1 \land t_2)|
=   |\{t_1 \land t_2\} \cup tm(t_1) \cup tm(t_2)|
=   |\{t_1 \land t_2, t_1, t_2\}|
=   3 \leq 3
=   1 + 1 + 1
=   1 + l(t_1) + l(t_2)
=   l(t_1 \land t_2)\\
$
\newpage
\subsection{vollstaendige Induktion}
\subsubsection\
I.A.:\\
$
n = 0\\
    \sum^0_{k=1}{\frac{1}{k * (k + 1)}} 
=   0
=   1 - 1
=   1 - \frac{1}{0+1}\\
$
\\
I.V.:\\
Die Aussage gelte fuer bel., aber feste $n \in \mathbb{N}$\\
\\
I.S.:\\
$
z.z.:\\
\sum^{n+1}_{k=1}{\frac{1}{k * (k + 1)}} = 1 - \frac{1}{n+2}\\
\\
n \rightarrow n + 1\\
            \sum^{n+1}_{k=1}{\frac{1}{k * (k + 1)}} 
=           \sum^{n}_{k=1}{\frac{1}{k * (k + 1)}} + \frac{1}{(n+1)*(n+2)}\\
=^{I.V.}    1 - \frac{1}{n+1} + \frac{1}{(n+1)*(n+2)}\\
=           1 - \frac{1}{(n+1)*(n+2)}(n+2-(1))\\
=           1 - \frac{n+1}{(n+1)*(n+2)}\\
=           1 - \frac{1}{n+2}\\
$
\subsubsection\
I.A.:\\
n = 0\\
$
    \sum^0_{k=0}{k^3}
=   0
=   0 * (0 + 1)^2
=   \frac{1}{4} * 0 ^2 * (0 + 1)^2\\
$
\\
I.V.:\\
Die Aussage gelte fuer bel., aber feste $n \in \mathbb{N}$\\
\\
I.S.:\\
$
z.z.:\\
    \sum^{n+1}_{k=0}{k^3} = \frac{1}{4} * (n+1) ^2 * (n + 2)^2\\
\\
n \rightarrow n+1\\
            \sum^{n+1}_{k=0}{k^3}\\
=           \sum^n_{k=0}{k^3} + (n+1)^3\\
=^{I.V.}    \frac{1}{4} * n^2 * (n+1)^2 + (n+1)^3\\
=           \frac{1}{4} * (n+1)^2 * (n^2 + 4 * (n+1))\\
=           \frac{1}{4} * (n+1)^2 * (n^2 + 4n + 4))\\
=           \frac{1}{4} * (n+1)^2 * (n+2)^2\\
$
\subsection{verallgemeinerte Induktion, vollstaendige Induktion}
\subsubsection\
I.A.:\\
fuer $a_i, i \in {0,1,2}$\\
$
a_0 = 0 = fib(0)\\
a_1 = 1 = fib(1)\\
a_2 = 1 = fib(0) + fib(1) = fib(2)\\
$
\\
I.V.:\\
Die Aussage gelte fuer alle Vorgaenge von $n \leq 3, n \in \mathbb{N}$\\
\\
I.S.:\\
$
    a_n
=   \frac{1}{2} * a_{n-3} + \frac{1}{2} * 3 * a_{n-2} + \frac{1}{2} * a_{n-1}
=   \frac{1}{2} * (a_{n-3} + 3 * a_{n-2} + a_{n-1})\\
=   \frac{1}{2} * (a_{n-3} + 2 * a_{n-2} + fib(n-2) + fib(n-1))\\
=   \frac{1}{2} * (a_{n-3} + a_{n-2} + a_{n-2} + fib(n))\\
=   \frac{1}{2} * (fib(n-3) + fib(n-2) + fib(n-2) + fib(n))\\
=   \frac{1}{2} * (fib(n-1) + fib(n-2) + fib (n))\\
=   \frac{1}{2} * (2 * fib(n))\\
=   fib(n)\\
$ 
\subsubsection\
I.A.:\\
$
n = 0\\
0^2 + 0 + 2 = 2 \leftarrow\ gerade\\
$
\\
I.V.:\\
Die Aussage gelte fuer bel., aber feste $n \in \mathbb{N}$\\
\\
I.S.:\\
$
z.z.:\\
(n+1)^2 + n + 3 \leftarrow gerade 
\\
n \rightarrow n + 1\\
    (n+1)^2 + n + 3
=   (n+1) * ((n+1) + 1 ) + 2\\
=   (n+1) * (n+2) + 2\\
=   (n^2 + 3n + 2) + 2\\
=   (n^2 + n + 2) + 2n + 2\\
$
Nach I.V. ist $(n^2 + n + 2)$ gerade. 
Eine moegliche Def. fuer gerade zahlen waere $G = \{2 * n | n \in \mathbb{N}\}$\\
Offensichtlich gilt $2n, 2 \in G$ 
und dass die Addition von nur geraden Zahlen eine gerade Zahl ergibt.\\
Damit gillt $\forall n \in \mathbb{N}.n^2 + n + 2\ ist\ gerade$\\
\newpage
\subsection{Teilbarkeitsrelation}
\subsubsection\
- reflexiv, da $\forall x \in \mathbb{N}. x | x$ (k = 0)\\
\\
-transitiv, da $\forall x,y,z \in \mathbb{N}.x | y \land y | z \Rightarrow x | z$ (x * k = z)\\
\\
noch zu zeigen:\\
Teilbarkeitsrelation ist noethersch(!)\\
bzw. jede nichtleere Teilmenge bzgl. "I" ein minimales Element besitzt.\\
\\
Sei $\emptyset \neq A \subseteq \mathbb{N}$\\
Ein minimales Element A ist ein Element, das keine Teiler hat.\\
0 wird von jeder Zahl $n \in \mathbb{N}$ geteilt.\\
daher kann 0 nur ein min. Element sein, wenn $A = \{0\}$\\
hat A mehr Elemente, so ist das naechst kleinste Element das minimale Element.\\
\\
Beweis durch Widespruch:\\
$
A = \{0,n,..\}\\
$Annahme n sei Teiler:$\\
\exists m \in A. m | n\\
(i)\ m = 0\\
0 | n,\ bzw.\ \frac{n}{0} \lightning\\
(ii) m = n\\
n | n,\ bzw.\ \frac{n}{n} \checkmark\\
(iii) m > n\\
\lightning,\ da\ m \nleq n\\
$
Somit auch noetherschhe Ordnung\\
\subsubsection\

\end{document}
