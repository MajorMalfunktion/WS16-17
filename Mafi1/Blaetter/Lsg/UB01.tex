\documentclass{article}
\author{Max Springenberg}
\title{Uebungsgblatt 1}
\setcounter{section}{1}
\usepackage{amssymb}

\begin{document}
\maketitle
\newpage

\subsection{Aussagenlogik Quizshow}
\subsubsection\ 
A:\\
1: Die Box ist leer\\
2: Das Gold ist nicht in dieser Box\\
3: Das Gold ist in Box 2\\
\\
A = $
    (B_1 \land \neg B_2 \land \neg B_3) \lor
    (\neg B_1 \land B_2 \land \neg B_3) \lor
    (\neg B_1 \land \neg B_2 \land B_3)
    $\\
\\
$C_i :=$ Hinweis i ist richtig\\
\\
$
C_1 := \neg B_1 \land B_2 \land \neg B_2 
\equiv \bot\\ 
C_2 := B_1 \land \neg B_2 \land \neg B_2 
\equiv B_1 \land \neg B_2 \\
C_3 := B_1 \land B_2 \land B_2i 
\equiv B_1 \land B_2\\
$
\\
\subsubsection\ 
Wenn in einer Spalte $A \land (C_1 \lor C_2 \lor C_3) \equiv A \land (C_2 \lor C_3)$ wahr ist, 
so ist in dieser die Wahrheitsbelegung der Boxen die richtige.\\
\\
\begin{tabular}{llllll}
        $B_1$   &$B_2$  &$B_3$  &A  &$(C_2 \lor C_3)$   &$A \land (C_2 \lor C_3)$\\
        0       &0      &1      &1  &0                  &0\\
        0       &1      &0      &1  &0                  &0\\
        0       &1      &1      &0  &0                  &0\\
        1       &0      &0      &1  &1                  &1\\
        1       &0      &1      &0  &1                  &0\\
        1       &1      &0      &0  &1                  &0\\
        1       &1      &1      &0  &1                  &0\\
\end{tabular}\\
\\
damit ist das Gold in Box 1.\\
\\
\subsection{logische Umformungen} 
\subsubsection\
Neutralitaet:\\
$
        \top \land A 
\equiv  (A \lor \neg A) \land A 
\equiv  A\\
        F \lor A 
\equiv  (A \land \neg A) \lor A
\equiv  A\\
$
\subsubsection\
Idempotenz:\\
$
        A \lor A
\equiv  (A \lor A) \land \top
\equiv  (A \lor A) \land (A \lor \neg A)
\equiv  A \lor (A \land \neg A)
\equiv  A \lor F
\equiv  A\\
        A \land A
\equiv  (A \land A) \lor \top
\equiv  (A \land A) \lor (A \land \neg A)
\equiv  A \land (A \lor \neg A)
\equiv  A \land \top
\equiv  A\\
$
\subsubsection\
Doppelnegation:\\
$
        \neg \neg A
\equiv  \neg \neg A \land \top
\equiv  \neg \neg A \land (A \lor \neg A)
\equiv  (\neg \neg A \land A) \lor (\neg \neg A \land \neg A)
\equiv  (\neg \neg A \land A) \lor F
\equiv  (\neg \neg A \land A) \lor (A \land \neg A)
\equiv  A \land (A \lor \neg A)
\equiv  A \lor F 
\equiv  A\\
$
\subsection{Mengen und Mengenaequivalenzen}
\subsubsection\
(a)\\
$
    (A^C \cup B^C)^C \cup (A^C \cup B^C)^C
=   (A \cap B) \cup (A \cap B^C)
=   A \cap (B \cup B^C)
=   A \cap M 
=   A\\
$
\\
(b)\\
$
    ((A \cup B)^C \cap E)^C \cup (D \cap A)
=   (A^C \cap B^C \cap E)^C \cup (D \cap A)
=   A \cup B \cup E^C \cup (D \cap A)
=   A \cup B \cup E^C
$
\subsubsection\
"$\Rightarrow$":\\
$
A = B \Rightarrow P(A) = P(B)\\
P(A) = \{x | x \subseteq A\} = \{x | x \subseteq B\} = P(B)\\
$
\\
"$\Leftarrow$":\\
Kontraposition bzgl. $A \neq B \Rightarrow P(A) \neq P(B)$\\
Sei $A \neq B$ und damit $a \in A \setminus B, \{a\} \subseteq A$\\
damit gilt $\{a\} \subseteq P(A), \{a\} \nsubseteq P(B)$\\
und damit auch$P(A) \neq P(B)$\\
\subsection{Mengen}
\subsubsection\
$A \cap B \neq \emptyset$:\\
gelte $\exists x. x \in A \Rightarrow x \in B$\\
nach Df. der Mengendifferenz gilt $x \notin ((A \setminus B) \cup (B \setminus A))$\\
es gilt ferner $x \in A \cup B \Rightarrow (A \setminus B) \cup (B \setminus A) \neq A \cup B$\\
\subsubsection\
Gegenbeispiel:\\
$
A = B = \emptyset\\
\\
    P(\emptyset) 
=   \{ \emptyset \}
\neq P(A) \subset P(B) 
=   P( \emptyset ) \setminus P( \emptyset )
= \emptyset
$
\end{document}
