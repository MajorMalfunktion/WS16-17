\documentclass{article}
\title{MafI 1 UB07}
\usepackage{amsmath}
\usepackage{amssymb}
\usepackage{stmaryrd}
\usepackage[onehalfspacing]{setspace}

\onehalfspacing
\setcounter{section}{12}
\begin{document}
\maketitle
\newpage

\subsection{Invertieren von matrizen}
\subsubsection\
$
\begin{pmatrix}
        1 &-1& 2&|& 1& 0& 0\\
        -3& 2& 4&|& 0& 1& 0\\
        2 &-3& 9&|& 0& 0& 1\\
\end{pmatrix}\\
\begin{pmatrix}
        1 &-1& 2&|& 1& 0& 0\\
        0 &-1&10&|& 3& 1& 0\\
        0 &-1& 5&|&-2& 0& 1\\
\end{pmatrix}\\
\begin{pmatrix}
        1 &-1& 2&|& 1& 0& 0\\
        0 &-1&10&|& 3& 1& 0\\
        0 & 0&-5&|&-5&-1& 1\\
\end{pmatrix}\\
\begin{pmatrix}
        1 &-1& 2&|& 1& 0& 0\\
        0 &-1& 0&|&-7&-1& 2\\
        0 & 0&-5&|&-5&-1& 1\\
\end{pmatrix}\\
\begin{pmatrix}
        1 &-1& 2&|& 1& 0& 0\\
        0 & 1& 0&|& 7& 1&-2\\
        0 & 0& 1&|& 1&\frac{1}{5}&\frac{-1}{5}\\
\end{pmatrix}\\
\begin{pmatrix}
        1 & 0& 0&|& 6&\frac{3}{5}&-\frac{12}{5}\\
        0 & 1& 0&|& 7& 1&-2\\
        0 & 0& 1&|& 1&\frac{1}{5}&\frac{-1}{5}\\
\end{pmatrix}\\
$
\\
damit ist 
$A^{-1} = 
\begin{pmatrix}
        6&\frac{3}{5}&-\frac{12}{5}\\
        7& 1&-2\\
        1&\frac{1}{5}&\frac{-1}{5}\\
\end{pmatrix}\\
$\\
\subsubsection\
$
A' = 
\begin{pmatrix}
        1 &-1& 2\\
        -3& 2& 4\\
        2 &-3& 4\\
\end{pmatrix}\\
\\
det(A') =    (1*2*4) + (-1*4*2) + (2*3^2) 
            -(2^3) - (-3*4*1) - (4*-3*-1)\\
        =   8 - 8 + 18 - 8 + 12 - 12\\
        =   10 \neq 0\\
$
\\
damit ist $A'$ invertierbar\\
\\
\\
z.z.: $A'$ auch ueber $\mathbb{Z}_5$ invertierbar\\
$
\begin{pmatrix}
        1 &-1& 2&|& 1& 0& 0\\
        -3& 2& 4&|& 0& 1& 0\\
        2 &-3& 4&|& 0& 0& 1\\
\end{pmatrix}\\
\begin{pmatrix}
        1 &-1& 2&|& 1& 0& 0\\
        0 &-1&10&|& 3& 1& 0\\
        0 &-1& 0&|&-2& 0& 1\\
\end{pmatrix}\\
\begin{pmatrix}
        1 & 0& 2&|& 3& 0&-1\\
        0 & 0&10&|& 5& 1&-1\\
        0 & 1& 0&|& 2& 0&-1\\
\end{pmatrix}\\
\begin{pmatrix}
        1 & 0& 2&|& 3& 0&-1\\
        0 & 0& 1&|&\frac{1}{2}&\frac{1}{10}&\frac{-1}{10}\\
        0 & 1& 0&|& 2& 0&-1\\
\end{pmatrix}\\
\\
\{\frac{1}{2},\frac{1}{10},\frac{-1}{10}\} \notin \mathbb{Z}_5\\
$
damit existiert keine Inverse zu $A'$ ueber $\mathbb{Z}_5$\\
\newpage
\subsection{Lineare Abbildungen}
\subsubsection\
a)\\
Gegenbeispiel:\\
$
v = (1,1,1)^t,\ \ w = (0,0,1)^t\\
f(v+w)  = f(1,1,2)^t\\
        = (3,3,2)^t \neq (3,3,1)^t\\
        = (3,3,1)^t + (0,0,0)^t\\
        = f(v) + f(w)\\
$
\\
b)\\
$
f(v+w)  = (-l(v+w), (l(v+w) + l(2*(v+w))))\\
        = ((-l(v) + -l(w)), (l(2v) + l(2w)))\\
        = (-l(v), l(2v)) + (-l(w), l(2w))\\
        = f(v) + f(w)\\
\\
f(s*v)  = (-l(s*v), (l(s*v) + l(s*2*v)))\\
        = (s*-l(s*v), (s*l(v) + s*l(2*v)))\\
        = s * f(v)\\
$
\subsubsection\

\newpage
\subsection{Basistransformation}
\subsubsection\
$
    [\varphi]_{E_3} 
=   \{
        (1,2)^t, (2,3)^t, (3,4)^t
    \}\\
    _{E_2}[\varphi]_{E_3}
=   \{
        (1,2)^t, (2,3)^t, (3,4)^t
    \}\\
$
\subsubsection\
$
B_3 =   \{
            (1,2,-1)^t, (2,-1,2)^t, (3,1,-1)^t
        \},\ 
B_2 =   \{
            (1,2)^t, (2,3)^t
        \}\\
\   [\varphi]_{B_3}
=   \{
        (2,4)^t, (6,9)^t, (2,5)^t
    \}\\
    _{B_2}[\varphi]_{B_3}
=   \{
        (2,0)^t, (0,3)^t, (4,-1)^t
    \}\\
$
\newpage
\subsection{Inversen von Matrizen}
\subsubsection\
$
A, B \in K^n\\
A =
\begin{pmatrix}
        a_{11}&...&a_{1n}\\
        ...&...&...\\
        a_{m1}&...&a_{mn}\\
\end{pmatrix}\\
B =
\begin{pmatrix}
        b_{11}&...&b_{1n}\\
        ...&...&...\\
        b_{m1}&...&b_{mn}\\
\end{pmatrix}\\
A * B =
\begin{pmatrix}
        a_{11} * b_{11}&...&a_{1n} * b_{m1}\\
        ...&...&...\\
        a_{m1} * b_{1n}&...&a_{mn} * b_{mn}\\
\end{pmatrix}\\
B * A =
\begin{pmatrix}
        b_{11} * a_{11}&...&b_{1n} * a_{m1}\\
        ...&...&...\\
        b_{m1} * a_{1n}&...&b_{mn} * a_{mn}\\
\end{pmatrix}\\
$
\\
Folglich gilt, dass wenn\\
$
    \begin{pmatrix}
            a_{11} * b_{11}&...&a_{1n} * b_{m1}\\
    \end{pmatrix}\\
=   \begin{pmatrix}
            1&...&0\\
    \end{pmatrix}\\
$\\
\\dann auch\\$
    \begin{pmatrix}
            b_{11} * a_{11}
            ...\\
            b_{m1} * a_{1n}
    \end{pmatrix}\\
=   \begin{pmatrix}
            1&...&0\\
    \end{pmatrix}^t\\
$
etc. ...\\
\\
Es findet lediglich eine Spiegelung an der Diagonalen statt.\\
Da diese in $E^{n \times n}$ als einzige mit Werten initialisiert ist,
gilt allgemeingueltig $A^n * B^n = B^n * A^n$, wenn $A^n * B^n = E^{n \times n}$
\subsubsection\
$
A =
\begin{pmatrix}
        1&0&0\\
        0&1&0\\
\end{pmatrix}\\
B = 
\begin{pmatrix}
        1&0\\
        0&1\\
        a&b\\
\end{pmatrix}\\
a, b \in K\\
A \in K^{2 \times 3}, B \in K^{3 \times 2}\\
A * B = E_2\\
B * A =
\begin{pmatrix}
        1&0&0\\
        0&1&0\\
        a&b&0\\
\end{pmatrix} \neq E_3\\
$
\end{document}
